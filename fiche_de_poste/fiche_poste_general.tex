\documentclass[12pt]{article}

%LES PACKAGES UTILISÉS
\usepackage{xcolor} %pour les couleurs
\usepackage[utf8]{inputenc} %pour l'écriture
\usepackage[french]{babel} %pour avoir les textes automatiques en français
\usepackage{graphicx} %pour les graphes
\graphicspath{{%the path to the logo_EJE picure}}
\usepackage{fancyhdr} %pour le formatage des pages
\usepackage[left=2cm,right=2cm,top=2cm,bottom=2.5cm]{geometry} %formatage du document
\usepackage{tikz}
\usepackage{flowfram} % Required for the multi-column layout
\usepackage{multicol}



%LES COULEURS PROPRES A EJE
\definecolor{rouge_eje}{RGB}{155,27,34}

%LE FORMATAGE DES PAGES
\pagestyle{fancy}
\fancyhf{}
\lhead{ENSAE Junior Études}
\rhead{Fiche de poste - [poste]}
\cfoot{\thepage}

\newcommand{\ejeBullet}{\textcolor{rouge_eje}{$\circ$}} % New command for the blue bullets

\newcommand{\ejeBulletFull}{\textcolor{rouge_eje}{•}}

\begin{document}



\begin{tikzpicture}
\node [shift={(4cm, 0cm)}] at (current page.north west)
        { \includegraphics[height = 2.5cm, width = 2.5cm]{logo_EJE.png} };
       
\node [shift = {(8cm, 0cm)}, xshift=3 cm , yshift = 0.5cm] at (current page.north west) {\Huge{\textbf{Fiche de poste}}};
\node [shift = {(8cm, 0cm)} ,xshift = 3cm, yshift = -0.5cm] at (current page.north west){\Huge{\textbf{Responsable Audits}}};

\end{tikzpicture}

\begin{tikzpicture}
\draw[color = rouge_eje, ultra thick](0,1) -- (16.5, 1);
\end{tikzpicture}



\begin{center}
Relative à l'évolution du poste de [poste]
\end{center}

\textbf{Position:}

[Description du poste, ce qu'il fait au sein de EJE, à quoi il sert].
\bigskip

\textbf{Temps de travail:}

Le temps de travail estimé pour ce poste est d'environ [nombre d'heures en moyenne pendant la semaine], mais comme plusieurs tâches sont ponctuelles, il peut varier en fonction des périodes.
\bigskip

\textbf{Mission:}
\begin{itemize} \itemsep5pt
\item[\ejeBulletFull] [un point général 1]
	\begin{itemize}
	\item[\ejeBullet] [détail sur ce point 1]
\item[\ejeBullet] [détail sur ce point 2]
	\end{itemize}
\item[\ejeBulletFull] [un point général 2]
	\begin{itemize}
	\item[\ejeBullet] [un détail sur ce point 1]
	\item[\ejeBullet] [un point sur ce détail 2] 


\bigskip

\textbf{Compétences:}
\begin{itemize}
\item[\ejeBulletFull] [compétence 1]
\item[\ejeBulletFull] [compétence 2]
\item[\ejeBulletFull] [compétence 3...]

\end{itemize}

\newpage

\large Ce document décrit les exigences du poste de [poste] au sein d'ENSAE Junior Études, pour donner une vision aux postulants
\\
\\
Edition [année de l'édition]: [auteur], [poste] ENSAE Junior Études
\\
\\
\textbf{\textcolor{rouge_eje}{Suivi des mises à jour}}

\begin{center}
\begin{tabular}{|p{2.5cm} | p{5cm} | p{5cm}|p{5cm}|}
\hline
Date & Auteur & Poste & Nature de la modification \\
\hline
[date de maj 1] & [auteur de maj] & [poste] & [infos sur maj] \\
\hline
[date de maj 2] & [auteur de maj 2] & [poste] & [infos sur maj] \\
\hline

\end{tabular}
\end{center}









\end{document}
