\documentclass[12pt]{article}

%LES PACKAGES UTILISÉS
\usepackage{xcolor} %pour les couleurs
\usepackage[utf8]{inputenc} %pour l'écriture
\usepackage[french]{babel} %pour avoir les textes automatiques en français
\usepackage{graphicx} %pour les graphes
\graphicspath{%add your own graphicspath to logo_EJE}
\usepackage{fancyhdr} %pour le formatage des pages
\usepackage[left=2cm,right=2cm,top=2cm,bottom=2.5cm]{geometry} %formatage du document
\usepackage{tikz}
\usepackage{flowfram} % Required for the multi-column layout
\usepackage{multicol}



%LES COULEURS PROPRES A EJE
\definecolor{rouge_eje}{RGB}{155,27,34}

%LE FORMATAGE DES PAGES
\pagestyle{fancy}
\fancyhf{}
\lhead{ENSAE Junior Études}
\rhead{Processus - [nom du processus]}
\cfoot{\thepage}

\begin{document}

\begin{tikzpicture}
\node [shift={(4cm, 0cm)}] at (current page.north west)
        { \includegraphics[height = 2.5cm, width = 2.5cm]{logo_EJE.png} };
       
\node [shift = {(8cm, 0cm)}, xshift=3 cm , yshift = 0.5cm] at (current page.north west) {\Huge{\textbf{Fiche de processus}}};
\node [shift = {(8cm, 0cm)} ,xshift = 3cm, yshift = -0.5cm] at (current page.north west){\Huge{\textbf{[nom du processus]}}};

\end{tikzpicture}

\begin{tikzpicture}
\draw[color = rouge_eje, ultra thick](0,1) -- (16.5, 1);
\end{tikzpicture}

\bigskip

\underline{Mandat:}\\

[année du mandat]
\\

\underline{Contexte du processus:}\\

Le [processus] s'inscrit dans l'activité de EJE. Il sert à EJE bla bla...
\\
Description en gros de ce que le processus fait dans la JE
\\

\underline{Responsable du processus:}
\\

[admin(s) responsable(s) du processus, poste, pôle]


\section*{Description du processus}


\subsection*{Sources d'éléments d'entrée}
D'où viennent les éléments d'entrée?
\subsection*{Eléments d'entrée}
 Les éléments d'entrée
 
 \subsection*{Procédures/Activités}
\noindent
\textbf{Procédures:}\\
Quelles actions mises en place pour le processus? Mettre le nom des actions, la description viendra plus tard
\\
\\
\textbf{Exigences:}\\
Quelles sont les exigences qu'il faut respecter quand on fait ce processus (exigences de EJE, des autres prestataires etc.)
\\
\\
\textbf{Documents du processus (si applicable)}\\
Quels sont les documents qui relatifs á ce processus?


\subsection*{Eléments de sortie}

Quelle est la sortie de nos processus?

\subsection*{Destinataires des éléments de sortie}

Vers qui vont nos éléments de sortie?


\newpage

\section*{Description des procédures}

Description litéraire ou schématique des procédures qui composent le processus

\section*{Indicateurs}

Mettre les indicateurs de ce processus
\\
\textit{TODO: repertorier tous les indicateurs et leurs donner un nom en Q.IN}

\section*{Enregistrement}

Où est ce qu'on enregistre les données relatives à ce processus.

\newpage

\large Ce document est un document type de fiche de processus dans le cadres des exigences de la norme ISO 9001 au sein de ENSAE Junior Études
\\
\\
Edition 2021/2022: Bastian Dupoirieux, responsable audits ENSAE Junior Études
\\
\\
\textbf{\textcolor{rouge_eje}{Suivi des mises à jour}}

\begin{center}
\begin{tabular}{|p{2.5cm} | p{5cm} | p{5cm}|p{5cm}|}
\hline
Date & Auteur & Poste & Nature de la modification \\
\hline
28/10/2021 & Bastian Dupoirieux & Responsable Audits & Première version du document \\
\hline

\end{tabular}
\end{center}




\end{document}